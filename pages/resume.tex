\begin{paracol}{2}
%---------------------------------------------------------------------------------------
%	LEFT COLUMN
%----------------------------------------------------------------------------------------
\begin{leftcolumn}

% Profileimage
\cutpic{0cm}{\linewidth}{assets/images/profile_picture.jpeg}

% Contact Section
\cvsection{Kontakt}

\icontext{Phone}{14}{\href{tel:+491737033077}{\telephone}}{black}\\[6pt]
\iconemail{EnvelopeSquare}{14}{\mail}{\mail}{black}\\[6pt]
\icontext{MapMarker}{14}{\href{https://maps.app.goo.gl/1FTs2EGCNSdXyY987}{\address}}{black}\\[6pt]
\icontext{BirthdayCake}{14}{\href{https://www.onthisday.com/date/1998/july/19}{19. July 1998, Köln}}{black}\\[6pt]
\icontext{Linkedin}{14}{\href{https://www.linkedin.com/in/joel-maximilian-mai/}{joel-maximilian-mai}}{black}\\[6pt]
\icontext{Xing}{14}{\href{https://www.xing.com/profile/JoelMaximilian_Mai/portfolio}{JoelMaximilian$\_$Mai}}{black}\\[6pt]
\icontext{MousePointer}{14}{\href{https://www.maispace.de}{www.maispace.de}}{black}\\[6pt]
\icontext{Github}{14}{\href{https://github.com/mai-space}{mai-space}}{black}\\[6pt]

% Skill Section
\cvsection{Fähigkeiten}

\cvskill{TYPO3} {\number\numexpr\year-2020\relax+ Jahre} {0.9} \\[-2pt]

\cvskill{WordPress} {\number\numexpr\year-2023\relax+ Jahre} {0.5} \\[-2pt]

\cvskill{Symfony} {\number\numexpr\year-2023\relax+ Jahre} {0.7} \\[-2pt]

\cvskill{VueJs} {\number\numexpr\year-2015\relax+ Jahre} {0.85} \\[-2pt]

\cvskill{Python} {\number\numexpr\year-2023\relax+ Jahre} {0.8} \\[-2pt]

\cvskill{Kotlin} {\number\numexpr\year-2014\relax+ Jahre} {0.9} \\[-2pt]

\cvskill{SQL} {\number\numexpr\year-2016\relax+ Jahre} {0.7} \\[-2pt]

\cvskill{No-SQL} {\number\numexpr\year-2023\relax+ Jahre} {0.4} \\[-2pt]

\end{leftcolumn}

%---------------------------------------------------------------------------------------
%	RIGHT COLUMN
%----------------------------------------------------------------------------------------
\begin{rightcolumn}

% Header
\fcolorbox{white}{primary}{\begin{minipage}[c][3.5cm][c]{1\mpwidth}
	\begin {center}
		\HUGE{ \textbf{ \textcolor{white}{ \uppercase{ \myname } } } } \\[-24pt]
		\textcolor{white}{ \rule{0.1\textwidth}{1.25pt} } \\[4pt]
		\large{ \textcolor{white} {Angestrebte Position: \jobposition} }
	\end {center}
\end{minipage}} \\[14pt]
\vspace{-12pt}

% Work Experience
\cvsectionAccent{Berufliche Erfahrung}

\cveventAccent
	{seit April 23}
	{Junior Fullstack Entwickler}
	{IW Medien GmbH}
	{
		In dieser Rolle war ich für die Entwicklung und Wartung von Web-Projekten verantwortlich. Ich habe PHP-Bibliotheken und Extensions mit Symfony und Extbase entwickelt, Jira und agiles Arbeiten vorangetrieben sowie Webseiten mit TYPO3, WordPress und API-Schnittstellen mit asynchronem JavaScript implementiert.
	}
	{PHP, TypoScript, Symfony, Extbase, TypeScript, SCSS, DevOps, TYPO3, HTML, CSS, JavaScript, Scrum, Kanban, Jira}
\vfill\null

\vspace{6pt}

\cvevent
	{2020 – 2023}
	{Werkstudent}
	{IW Medien GmbH}
	{
		Als Werkstudent habe ich schnell die Verantwortung für kleinere Projekte übernommen, einschließlich der Entwicklung von kleineren Webseiten und Newslettern. Während dieser Zeit habe ich SCSS, JavaScript, Fluid und Extbase gelernt und angewendet.
	}
	{TYPO3, Extbase, Fluid, HTML, CSS, JavaScript, Newsletter, DevOps, Deployment, SCSS}
\vfill\null

% Education Section
\cvsection{Ausbildung}

\cvevent
	{2016 - 2023}
	{Bachelor - Medieninformatik}
	{Technische Hochschule Köln}
	{
		Das Studium der Medieninformatik war für mich eine äußerst bereichernde Zeit, in der ich meine Leidenschaft für Technik und Design vertiefen konnte. Ich habe fundierte Kenntnisse in VueJs, objektorientierter Programmierung mit Java sowie in verschiedenen Design- und Code-Patterns erworben.
	}
	{Abschlussnote: \textbf{2,3}. Bachelorarbeit: \textbf{1,0}.}
\vfill\null

\vspace{6pt}

\cvevent
	{2008 - 2016}
	{Abitur - Mathematik und Informatik Leistungskurs}
	{Gymnasium Kerpen (Europaschule)}
	{
		Während meines Abiturs habe ich meine Liebe zur Informatik entdeckt. Besonders die Leistungskurse in Mathematik und Informatik haben mein Interesse geweckt, und ich habe grundlegende Kenntnisse in Java und JavaScript erlangt.
	}
	{Abschlussnote: \textbf{2,3}.}
\vfill\null

% Make Column Border full length 
\vfill
\vfill
\vfill
\end{rightcolumn}
\end{paracol}